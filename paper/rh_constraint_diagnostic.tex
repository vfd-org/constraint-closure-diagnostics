% RH Constraint-Diagnostic Paper
% A Constraint-Closure Diagnostic Framework with Application to RH-Motivated Spectral Structures
%
% IMPORTANT: This paper does NOT claim a proof of the Riemann Hypothesis.
% It presents a diagnostic framework and case study with falsifiable results.

\documentclass[11pt,a4paper]{article}

% Packages
\usepackage[utf8]{inputenc}
\usepackage[T1]{fontenc}
\usepackage{amsmath,amssymb,amsthm}
\usepackage{graphicx}
\usepackage{booktabs}
\usepackage{hyperref}
\usepackage{xcolor}
\usepackage{geometry}
\usepackage{float}
\usepackage{caption}
\usepackage{subcaption}
\usepackage{algorithm}
\usepackage{algpseudocode}
\usepackage{listings}
\usepackage{natbib}

% Page geometry
\geometry{margin=1in}

% Hyperref setup
\hypersetup{
    colorlinks=true,
    linkcolor=blue,
    citecolor=blue,
    urlcolor=blue
}

% Theorem environments - using Proposition and Observation, not Theorem
\newtheorem{proposition}{Proposition}[section]
\newtheorem{observation}[proposition]{Observation}
\newtheorem{definition}[proposition]{Definition}
\newtheorem{remark}[proposition]{Remark}

% Custom commands
\newcommand{\VFD}{\textsf{VFD}}
\newcommand{\BA}{\textsf{BA}}
\newcommand{\BN}{\textsf{BN}}
\newcommand{\RH}{\textsf{RH}}

% Code listing style
\lstset{
    basicstyle=\ttfamily\small,
    breaklines=true,
    frame=single,
    language=bash
}

% Title
\title{A Constraint-Closure Diagnostic Framework\\with Application to RH-Motivated Spectral Structures}

\author{
    Lee Smart\\
    Vibrational Field Dynamics Institute\\
    \texttt{contact@vibrationalfielddynamics.org}\\
    \texttt{@vfd\_org}
}

\date{January 2026}

\begin{document}

\maketitle

% ============================================================================
% ABSTRACT
% ============================================================================

\begin{abstract}
We present a constraint-closure diagnostic framework for systematically evaluating
mathematical structures through hierarchical constraint satisfaction. The framework
is applied to a spectral case study motivated by the Riemann Hypothesis (RH),
where we construct a finite-dimensional operator system and evaluate its consistency
with prescribed structural properties.

The diagnostic system verifies closure through five hierarchical levels (L0--L4),
encompassing structural validity, explicit formula constraints, symmetry relations,
positivity conditions, and trace/moment consistency. All constraints are satisfied
to machine precision (residuals $< 10^{-14}$ in the production run and $< 10^{-13}$
across the $3 \times 3$ parameter sweep).

A bridge mapping projects spectral data to comparison points, achieving
rank correlation $\rho = 0.9997$ with reference values. Three distinct
negation tests (wrong ordering, wrong scale, wrong coordinate) all produce
measurably degraded results: ordering perturbation drops rank correlation
from $\rho = 0.9997$ to $\rho = 0.008$, scale perturbation increases RMSE
from 1098 to 2395, and coordinate perturbation increases $\beta$ deviation
from 0 to 0.2. This degradation under perturbation is consistent with
non-arbitrary structure.

\textbf{This paper does not claim a proof of RH.} We present a diagnostic framework,
empirical results, and falsifiability analysis. The observed correspondence is
inconsistent with simple null models (see Appendix~\ref{app:statistics}
for heuristic estimates). We discuss limitations, non-claims, and the
distinction between structural motivation and mathematical proof.

All results are reproducible via open-source software with fixed random seeds.
\end{abstract}

\vspace{1em}
\noindent\textbf{Keywords:} constraint satisfaction, spectral analysis, diagnostic framework,
falsifiability, reproducibility

% ============================================================================
% 1. INTRODUCTION
% ============================================================================

\section{Introduction}
\label{sec:introduction}

The Riemann Hypothesis (RH) remains one of the most important open problems in
mathematics. While extensive numerical verification has confirmed RH for the first
$10^{13}$ zeros \citep{odlyzko1989}, no proof has been found despite 150+ years
of effort.

We introduce a \emph{constraint-closure diagnostic framework}---a systematic
methodology for evaluating mathematical structures through hierarchical
constraint satisfaction. Unlike optimization-based approaches that seek to
minimize residuals, our framework constructs explicit objects and verifies
whether they satisfy prescribed constraints to within specified tolerances.

\subsection{Contributions}

This paper makes the following contributions:

\begin{enumerate}
    \item \textbf{Diagnostic Framework:} A five-level closure ladder (L0--L4)
    for systematic constraint verification, with clear failure modes at each level.

    \item \textbf{RH-Motivated Case Study:} Application to a spectral structure
    with positivity, symmetry, and trace bounds motivated by RH-related criteria.

    \item \textbf{Falsifiability Tests:} Three negation modes (BN1, BN2, BN3)
    that would detect spurious curve-fitting if present.

    \item \textbf{Reproducible Software:} Open-source implementation with
    deterministic execution and complete artifact export.
\end{enumerate}

\subsection{Scope and Limitations}
\label{subsec:scope}

\begin{quote}
\textbf{Important:} This paper is a diagnostic study, not a proof attempt.
We explicitly disclaim:
\begin{itemize}
    \item Any claim that RH is proven
    \item Any claim that our projections are actual zeta zeros
    \item Any claim that structural motivation implies logical equivalence
\end{itemize}
The framework verifies internal consistency of a mathematical construction.
Agreement with reference data is empirical, not proven.
\end{quote}

\subsection{AI-Readable Scope (Claims and Non-Claims)}
\label{subsec:ai_scope}

For automated analysis and verification systems:

\textbf{What is being shown:}
\begin{itemize}
    \item Constraint-closure ladder (L0--L4) passing to machine precision
    \item Reproducible deterministic execution with fixed seeds
    \item Falsification sensitivity: perturbations degrade correspondence
    \item Diagnostic evidence for non-arbitrary structure
\end{itemize}

\textbf{What is NOT being claimed:}
\begin{itemize}
    \item Proof of the Riemann Hypothesis
    \item Computation of actual zeta zeros
    \item Mathematical equivalence between VFD operators and $\zeta(s)$
    \item That structural motivation implies logical equivalence
\end{itemize}

\textbf{Interpretation:} These results support the view that RH-related spectral
behavior may be better understood as a property of a structured constraint system.
The correspondence between construction and reference data, combined with
degradation under perturbation, suggests that classical formulations may be
missing structural degrees of freedom. This is offered as a diagnostic hypothesis,
not a theorem.

\subsection{Paper Structure}

Section~\ref{sec:framework} presents the constraint-closure diagnostic framework.
Section~\ref{sec:case_study} describes the RH-motivated spectral case study.
Section~\ref{sec:results} presents closure and robustness results.
Section~\ref{sec:bridge} discusses the bridge mapping and its interpretation.
Section~\ref{sec:falsification} analyzes falsifiability and non-arbitrariness.
Section~\ref{sec:limitations} explicitly states limitations and non-claims.
Section~\ref{sec:reproducibility} covers software and reproducibility.
Section~\ref{sec:discussion} discusses implications.
Section~\ref{sec:conclusion} concludes.

% ============================================================================
% 2. CONSTRAINT-CLOSURE DIAGNOSTIC FRAMEWORK
% ============================================================================

\section{Constraint-Closure Diagnostic Framework}
\label{sec:framework}

\subsection{Motivation}

Traditional approaches to mathematical conjectures often proceed through:
(1) formulating a claim, (2) attempting to prove it, (3) reporting success or failure.
This binary outcome provides limited diagnostic information when proofs fail.

Our framework takes a different approach: construct explicit mathematical objects
and verify which constraints they satisfy. This provides:
\begin{itemize}
    \item Graduated feedback (which levels pass/fail)
    \item Quantitative residuals (how close to satisfaction)
    \item Robustness analysis (stability under parameter variation)
    \item Falsifiability (testable predictions)
\end{itemize}

\subsection{Closure Ladder Structure}

The diagnostic framework organizes constraints into five hierarchical levels,
evaluated sequentially with \emph{gating}: if level $L_k$ fails, levels
$L_{k+1}, \ldots, L_4$ are not evaluated.

\begin{table}[H]
\centering
\caption{Closure Ladder Levels}
\label{tab:closure_levels}
\begin{tabular}{@{}clll@{}}
\toprule
Level & Name & Constraint Families & Failure Mode \\
\midrule
L0 & Baseline & Structural validity & Malformed input \\
L1 & Algebraic Relations & Torsion order, Weyl relation & Algebraic inconsistency \\
L2 & Symmetry & Projector identities & Decomposition failure \\
L3 & Positivity & Kernel $K \geq 0$, $Q_K(v) \geq 0$ & Negative spectrum \\
L4 & Trace/Moment & Spectral moment consistency & Distributional mismatch \\
\bottomrule
\end{tabular}
\end{table}

\subsection{Constraint Families}

Each level comprises one or more constraint families:

\paragraph{Algebraic Relations (TW):} Verifies torsion and Weyl algebraic relations
(labeled ``EF'' in software output for historical reasons):
\begin{itemize}
    \item \textbf{T1}: Torsion periodicity $T^{12} = I$ (error $< 10^{-12}$)
    \item \textbf{W1}: Weyl relation $TST^{-1} = \omega S$ where $\omega = e^{2\pi i/12}$
\end{itemize}

\paragraph{Symmetry:} Verifies projector structure:
\begin{itemize}
    \item \textbf{P1}: Resolution of identity $\sum_q P_q = I$
    \item \textbf{P2}: Orthogonality $P_q P_r = \delta_{qr} P_q$
\end{itemize}

\paragraph{Positivity:} Verifies spectral properties:
\begin{itemize}
    \item \textbf{D1}: Self-adjointness $K = K^*$
    \item \textbf{D2}: Torsion commutation $[K, T] = 0$
    \item \textbf{D3}: Non-negativity $\lambda_{\min}(K) \geq 0$
\end{itemize}

\paragraph{Trace/Moment:} Verifies spectral distribution:
\begin{itemize}
    \item Moment consistency: $\mathbb{E}[\lambda^2] = 4R^2 + 2R$ (analytic formula)
\end{itemize}

\subsection{Residual Computation}

For each constraint family $\mathcal{F}$, we compute a residual:
\begin{equation}
r_{\mathcal{F}} = \max_{c \in \mathcal{F}} \| \text{actual}(c) - \text{expected}(c) \|
\end{equation}

A family is \emph{satisfied} if $r_{\mathcal{F}} < \tau$ where $\tau$ is the
family-specific tolerance (typically $10^{-8}$ to $10^{-12}$).

The total level residual is:
\begin{equation}
r_{L_k} = \sum_{\mathcal{F} \in L_k} r_{\mathcal{F}}
\end{equation}

% ============================================================================
% 3. RH-MOTIVATED SPECTRAL CASE STUDY
% ============================================================================

\section{RH-Motivated Spectral Case Study}
\label{sec:case_study}

\subsection{Construction Overview}

We construct a finite-dimensional operator system. The construction comprises:

\begin{enumerate}
    \item A state space $\mathcal{H} = \ell^2(\mathbb{Z}_C) \otimes \mathbb{C}^d$
    \item A torsion operator $T$ with $T^{12} = I$
    \item A shift operator $S$ satisfying the Weyl relation
    \item A kernel operator $K$ (graph Laplacian)
\end{enumerate}

The positivity and symmetry constraints are motivated by criteria appearing
in RH-related literature, but no logical equivalence is claimed or implied.

\subsection{State Space}

The state space has dimension $N = C \times d$ where:
\begin{itemize}
    \item $C$ = cell count (truncation parameter)
    \item $d$ = internal dimension (orbit structure)
\end{itemize}

For our production runs: $C = 64$, $d = 96$ (8 orbits of size 12), giving $N = 6144$.

\subsection{Torsion Operator}

The torsion operator $T$ acts diagonally with eigenvalues $\omega^q$ for
$q \in \{0, 1, \ldots, 11\}$, where $\omega = e^{2\pi i/12}$:
\begin{equation}
T |n, q\rangle = \omega^q |n, q\rangle
\end{equation}

This satisfies $T^{12} = I$ by construction.

\subsection{Kernel Operator}

The kernel $K$ is a graph Laplacian with propagation range $R$:
\begin{equation}
(K\psi)_n = 2R \cdot \psi_n - \sum_{d=1}^{R} (\psi_{n+d} + \psi_{n-d})
\end{equation}

The eigenvalues have the closed form:
\begin{equation}
\lambda_k = 4 \sum_{d=1}^{R} \sin^2\left(\frac{\pi k d}{N}\right)
\label{eq:eigenvalues}
\end{equation}

\begin{proposition}[Kernel Positivity of the Constructed Operator]
\label{prop:positivity}
For the kernel $K$ defined above, all eigenvalues $\lambda_k$ satisfy
$\lambda_k \geq 0$ for all $k$.
\end{proposition}

\begin{proof}
From Equation~\eqref{eq:eigenvalues}, $\lambda_k$ is a sum of squared sines:
$\lambda_k = 4 \sum_{d=1}^{R} \sin^2(\cdot) \geq 0$ since $\sin^2(\theta) \geq 0$
for all $\theta \in \mathbb{R}$.
\end{proof}

\begin{remark}
This positivity condition is motivated by positivity requirements appearing
in RH-related criteria (e.g., the Li criterion). However, this proposition
concerns only the constructed operator and does not imply or establish any
equivalence with those criteria.
\end{remark}

\subsection{Structural Properties}

The construction satisfies properties that are motivated by (but not logically
equivalent to) RH-related structures:

\begin{table}[H]
\centering
\caption{Structural Properties (Motivation, Not Equivalence)}
\label{tab:properties}
\begin{tabular}{@{}lll@{}}
\toprule
Property & In-Model Status & Motivation \\
\midrule
Kernel positivity ($\lambda_k \geq 0$) & Satisfied (Prop.~\ref{prop:positivity}) & Li criterion \\
Self-dual coordinate $\beta = 0.5$ & By construction & Critical line \\
Weyl relation & Verified ($< 10^{-10}$) & Functional equation \\
\bottomrule
\end{tabular}
\end{table}

\textbf{Note:} The ``Motivation'' column indicates conceptual inspiration only.
No logical implication or mathematical equivalence is claimed between the
in-model properties and the classical RH-related criteria.

% ============================================================================
% 4. RESULTS: CLOSURE AND ROBUSTNESS
% ============================================================================

\section{Results: Closure and Robustness}
\label{sec:results}

\subsection{Closure Ladder Results}

We present results from a production run with parameters:
\begin{itemize}
    \item Cell count $C = 64$
    \item Internal dimension $d = 96$
    \item Propagation range $R = 3$
    \item Random seed = 42
\end{itemize}

\begin{table}[H]
\centering
\caption{Closure Ladder Results (Run hash: \texttt{85568e827299b531})}
\label{tab:closure_results}
\begin{tabular}{@{}clcl@{}}
\toprule
Level & Status & Total Residual & Family Breakdown \\
\midrule
L0 & PASS & $0$ & (structural) \\
L1 & PASS & $1.35 \times 10^{-14}$ & TW: $1.35 \times 10^{-14}$ \\
L2 & PASS & $1.96 \times 10^{-14}$ & +Symmetry: $6.08 \times 10^{-15}$ \\
L3 & PASS & $1.96 \times 10^{-14}$ & +Positivity: $0$ \\
L4 & PASS & $1.96 \times 10^{-14}$ & +Trace: $0$ \\
\bottomrule
\end{tabular}
\end{table}

All residuals are below $2 \times 10^{-14}$, indicating machine-precision
satisfaction of all constraint families within this construction.

\begin{figure}[H]
\centering
\includegraphics[width=0.7\textwidth]{figures/fig01_residual_ladder.png}
\caption{Closure ladder residuals (L0--L4). All levels pass with residuals
$< 10^{-13}$. A failed level would show residual $> 10^{-8}$ (threshold marked).}
\label{fig:residual_ladder}
\end{figure}

\begin{figure}[H]
\centering
\includegraphics[width=0.7\textwidth]{figures/fig03_constraint_waterfall.png}
\caption{Constraint family waterfall showing residuals per family.
TW (Torsion/Weyl algebraic relations, labeled ``EF'' in software) dominates
at $\sim 10^{-14}$; Positivity and Trace contribute zero residual.}
\label{fig:constraint_waterfall}
\end{figure}

\subsection{Spectral Analysis}

\begin{figure}[H]
\centering
\includegraphics[width=0.8\textwidth]{figures/fig04_spectrum_histogram.png}
\caption{Kernel spectrum histogram. All 6144 eigenvalues are non-negative,
consistent with Proposition~\ref{prop:positivity}.
The leftmost bar at $\lambda = 0$ represents the zero mode(s).}
\label{fig:spectrum_histogram}
\end{figure}

\subsection{Parameter Robustness}

To verify that results are not artifacts of specific parameter choices,
we conducted a $3 \times 3$ parameter sweep over:
\begin{itemize}
    \item Cell count: $C \in \{16, 32, 64\}$
    \item Propagation range: $R \in \{1, 2, 3\}$
\end{itemize}

\begin{table}[H]
\centering
\caption{Parameter Sweep Results: Maximum Level Passed}
\label{tab:sweep_results}
\begin{tabular}{@{}c|ccc@{}}
\toprule
$C$ \textbackslash\ $R$ & 1 & 2 & 3 \\
\midrule
16 & L4 & L4 & L4 \\
32 & L4 & L4 & L4 \\
64 & L4 & L4 & L4 \\
\bottomrule
\end{tabular}
\end{table}

All 9 configurations pass all closure levels, with residuals consistently
$< 10^{-13}$.

\begin{figure}[H]
\centering
\includegraphics[width=0.7\textwidth]{figures/fig02_phase_map.png}
\caption{Phase map over parameter grid. Color indicates maximum level passed
(4 = all levels). All configurations achieve full closure.}
\label{fig:phase_map}
\end{figure}

\begin{figure}[H]
\centering
\includegraphics[width=0.7\textwidth]{figures/fig04_positivity_wall_grid.png}
\caption{Minimum eigenvalue over parameter grid. All values are
non-negative (positivity satisfied everywhere). Variation reflects
spectral structure, not constraint violation.}
\label{fig:positivity_wall}
\end{figure}

% ============================================================================
% 5. THE BRIDGE MAPPING
% ============================================================================

\section{The Bridge Mapping: Observable Behavior}
\label{sec:bridge}

\subsection{Motivation}

The closure ladder verifies internal consistency of the construction.
To assess correspondence with reference data, we introduce a
\emph{bridge mapping} that projects spectral data to comparison points.

\textbf{Important:} The bridge mapping is an external hypothesis, not part
of the core construction. Its validity is empirically tested, not proven.
The bridge implementation is included in the software repository for reproducibility;
we treat it as a testable hypothesis and evaluate it via falsification modes
rather than presenting it as a derived theorem.

\subsection{Projection Mechanism}

The bridge mapping is treated as an external, deterministic projection whose
internal construction is not required for interpreting the diagnostic results
presented here.

The mapping takes as input the kernel eigenvalues $\{\lambda_k\}$ and produces
projected values $\{t_k\}$ and coordinates $\{\beta_k\}$.

Observable properties of the projection:
\begin{itemize}
    \item \textbf{Monotonicity:} Larger eigenvalues map to larger projected values
    \item \textbf{Self-dual coordinate:} All $\beta_k = 0.5$ by construction
    \item \textbf{Determinism:} Identical inputs produce identical outputs
\end{itemize}

\subsection{Comparison with Reference Data}

Projected values are compared against reference zeta zeros computed via
\texttt{mpmath.zetazero()} with 50-digit precision.

\begin{table}[H]
\centering
\caption{Bridge Overlay Metrics ($n = 5000$ comparisons)}
\label{tab:bridge_metrics}
\begin{tabular}{@{}lr@{}}
\toprule
Metric & Value \\
\midrule
Pearson correlation & $0.99994$ \\
Spearman rank correlation & $0.99973$ \\
RMSE & $1098.0$ \\
Mean absolute error & $989.5$ \\
Self-dual $\beta$ & $0.5$ (exact) \\
$\beta$ deviation & $0.0$ \\
\bottomrule
\end{tabular}
\end{table}

\begin{figure}[H]
\centering
\includegraphics[width=0.9\textwidth]{figures/fig06_zero_overlay.png}
\caption{Zero overlay: projected spectrum (blue) vs.\ reference zeta zeros
(orange). The observed rank correlation is $\rho = 0.9997$. Systematic offset
reflects coordinate system difference.}
\label{fig:zero_overlay}
\end{figure}

\subsection{Interpretation}

We observe:
\begin{enumerate}
    \item Eigenvalue ordering maps to reference ordering with high fidelity
    \item The projection is sensitive to perturbation (see Section~\ref{sec:falsification})
\end{enumerate}

The systematic RMSE ($\sim 1000$) indicates that the construction and
reference data use different coordinate scales. The bridge translates
between these scales; it does not claim identity.

% ============================================================================
% 6. FALSIFICATION AND NON-ARBITRARINESS
% ============================================================================

\section{Falsification and Non-Arbitrariness}
\label{sec:falsification}

\subsection{The Falsifiability Challenge}

A critical question: \emph{Is the bridge mapping arbitrary?}

If the mapping were arbitrary, then perturbing it (using ``wrong'' mappings) should
produce similar results. We test this via three negation modes:

\begin{table}[H]
\centering
\caption{Bridge Negation Modes}
\label{tab:negation_modes}
\begin{tabular}{@{}clp{6cm}@{}}
\toprule
Mode & Perturbation & Expected Effect \\
\midrule
BN1 & Shuffle eigenvalue ordering & Destroy rank correlation \\
BN2 & Wrong scale factor ($\times 1.5$) & Increase RMSE \\
BN3 & Wrong $\beta$ offset (0.7 vs 0.5) & Non-zero $\beta$ deviation \\
\bottomrule
\end{tabular}
\end{table}

\subsection{Falsification Results}

\begin{table}[H]
\centering
\caption{Falsification Metrics (BA vs.\ Negations)}
\label{tab:falsification}
\begin{tabular}{@{}llccp{4cm}@{}}
\toprule
Mode & Metric & BA Value & BN Value & Effect \\
\midrule
BN1 & Spearman $\rho$ & $0.9997$ & $0.008$ & Ordering destroyed \\
BN2 & RMSE & 1098 & 2395 & $2.2\times$ worse \\
BN3 & $\beta$ deviation & $0$ (exact) & $0.2$ & Coordinate shifted \\
\bottomrule
\end{tabular}
\end{table}

All three negations produce measurably degraded results:
\begin{itemize}
    \item BN1 (wrong ordering): Shuffling eigenvalues drops rank correlation from $\rho = 0.9997$ to $\rho = 0.008$
    \item BN2 (wrong scale): Incorrect scale factor increases RMSE from 1098 to 2395 ($2.2\times$)
    \item BN3 (wrong coordinate): Wrong $\beta$ offset shifts deviation from 0 (exact) to 0.2
\end{itemize}

\begin{figure}[H]
\centering
\includegraphics[width=0.9\textwidth]{figures/fig07_falsification.png}
\caption{Falsification comparison: BA (left) vs.\ negations (right panels).
Each negation degrades specific metrics. The correspondence degrades under
perturbation.}
\label{fig:falsification}
\end{figure}

\subsection{Interpretation}

The falsification tests indicate:

\begin{enumerate}
    \item \textbf{Non-arbitrariness:} The mapping is not one of many
    equivalent choices. Perturbations degrade results.

    \item \textbf{Specificity:} Each perturbation targets a
    different structural property (ordering, scale, coordinate), and each
    causes specific degradation.

    \item \textbf{Sensitivity:} The correspondence is sensitive to the
    specific mapping used.
\end{enumerate}

\subsection{Statistical Considerations}

The falsification results are inconsistent with simple null models where
the mapping has no structural relationship to the reference data. Heuristic
probability estimates are provided in Appendix~\ref{app:statistics};
these are intended for scale intuition, not rigorous bounds.

% ============================================================================
% 7. LIMITATIONS AND NON-CLAIMS
% ============================================================================

\section{Limitations and Non-Claims}
\label{sec:limitations}

\subsection{What This Paper Does NOT Claim}

We explicitly disclaim the following:

\begin{enumerate}
    \item \textbf{No proof of RH.} The Riemann Hypothesis is not proven
    by this work. Internal consistency of a construction does not imply
    truth of an external conjecture.

    \item \textbf{No computation of zeta zeros.} The projected values are
    artifacts of the bridge mapping, not computed zeros of $\zeta(s)$.

    \item \textbf{No logical equivalence.} Structural motivation $\neq$
    mathematical equivalence. The construction is inspired by RH-related
    properties; it does not logically imply RH.

    \item \textbf{No uniqueness claim.} We do not claim this construction
    is the unique structure with these properties.

    \item \textbf{No derivation of bridge.} The bridge mapping is postulated,
    not derived from first principles.
\end{enumerate}

\subsection{Open Questions}

The following remain open:

\begin{enumerate}
    \item \textbf{Convergence:} Does correspondence improve as $N \to \infty$?

    \item \textbf{Uniqueness:} Are there other constructions with similar properties?

    \item \textbf{Bridge derivation:} Can the mapping be derived rather than postulated?

    \item \textbf{Logical gap:} What would close the gap between ``satisfies
    RH-motivated constraints'' and ``implies RH''?
\end{enumerate}

\subsection{Addressing Potential Concerns}

\paragraph{``Is this just curve-fitting?''}
Section~\ref{sec:falsification} shows that perturbing the mapping
(BN1, BN2, BN3) degrades results: BN1 drops rank correlation from $\rho = 0.9997$
to $\rho = 0.008$, BN2 increases RMSE from 1098 to 2395, and BN3 shifts $\beta$
deviation from 0 to 0.2.
Arbitrary curve-fitting would produce many equivalently good mappings.

\paragraph{``Is this a finite-size artifact?''}
The parameter sweep (Table~\ref{tab:sweep_results}) shows consistent results
across $C \in \{16, 32, 64\}$. All configurations pass all levels with
similar residuals. Behavior at larger $N$ remains an open question.

\paragraph{``Why should this mapping be unique?''}
We do not claim uniqueness. We observe that the tested mapping outperforms
the tested alternatives. Whether other good mappings exist is open.

\paragraph{``Where is the proof?''}
There is no proof of RH in this paper. We present a diagnostic framework
and case study with falsifiable results.

\paragraph{``Why should anyone care?''}
The framework documents: (1) a constructive approach to RH-motivated structures,
(2) falsifiability of proposed correspondences, (3) reproducible artifacts.
The observed correspondence is inconsistent with simple null models.

% ============================================================================
% 8. REPRODUCIBILITY AND SOFTWARE
% ============================================================================

\section{Reproducibility and Software}
\label{sec:reproducibility}

\subsection{Software Availability}

The diagnostic framework is implemented in Python and available at:
\begin{center}
\texttt{https://github.com/vfd-org/constraint-closure-diagnostics}
\end{center}

\subsection{System Requirements}

\begin{itemize}
    \item Python $\geq 3.8$
    \item NumPy $\geq 1.20$, SciPy $\geq 1.6$, mpmath $\geq 1.2$
    \item Memory: 4GB minimum, 8GB recommended
\end{itemize}

\subsection{Reproduction Commands}

All figures in this paper can be regenerated:

\begin{lstlisting}
# Install
pip install -e .

# Standard run (Figures 1-4)
rhdiag run --seed 42 --cell-count 64 --internal-dim 96

# Bridge run (Figures 6-7)
rhdiag run --seed 42 --cell-count 64 --internal-dim 96 --bridge-mode BA

# Parameter sweep (Figures 5-6)
rhdiag sweep --param1 cell_count --values1 16,32,64 \
             --param2 propagation_range --values2 1,2,3

# Complete bundle (all figures + reports)
rhdiag bundle --seed 42 --cell-count 64 --internal-dim 96
\end{lstlisting}

\subsection{Determinism}

All runs are deterministic given the random seed. Two runs with identical
parameters produce bit-identical outputs:

\begin{lstlisting}
# Verification
rhdiag run --seed 42  # Run 1: hash = abc123...
rhdiag run --seed 42  # Run 2: hash = abc123... (identical)
\end{lstlisting}

\subsection{Output Artifacts}

Each run produces:
\begin{itemize}
    \item \texttt{config.json}: Full configuration (replayable)
    \item \texttt{manifest.json}: Run metadata, tolerances, results
    \item \texttt{metrics.json}: All computed metrics
    \item \texttt{figures/*.png}: Visualization outputs
    \item \texttt{RELEASE\_REPORT.md}: Human-readable summary
\end{itemize}

\subsection{Paper Figure Synchronization}

The paper uses figures from \texttt{paper/figures/*.png}. After running
\texttt{rhdiag bundle}, figures are generated in \texttt{runs/.../figures/}.
To synchronize, use the provided script:

\begin{lstlisting}
# After rhdiag bundle, sync figures to paper directory:
python tools/sync_paper_figures.py --bundle-dir runs/release_<timestamp>

# The script automatically locates production run and sweep outputs,
# then copies all required figures to paper/figures/
\end{lstlisting}

The release package includes pre-generated figures in \texttt{paper/figures/}
that match the reference run hash \texttt{85568e827299b531}.

% ============================================================================
% 9. DISCUSSION
% ============================================================================

\section{Discussion}
\label{sec:discussion}

\subsection{The Diagnostic Paradigm}

Traditional approaches to RH attempt direct proof. Our framework takes a
different approach: construct explicit objects and verify properties. This provides:

\begin{enumerate}
    \item \textbf{Graduated feedback:} Know which constraints fail, not just
    ``proof failed.''

    \item \textbf{Falsifiability:} Testable predictions (negation modes) that
    could disprove the correspondence.

    \item \textbf{Reproducibility:} All results independently verifiable.
\end{enumerate}

\subsection{Observations}

We observe:

\begin{enumerate}
    \item \textbf{The construction satisfies RH-motivated constraints.} All five
    closure levels pass to machine precision.

    \item \textbf{The correspondence degrades under perturbation.} Falsification
    tests show measurable degradation.

    \item \textbf{Results are stable across parameters.} No finite-size artifacts
    detected in the tested range.
\end{enumerate}

\subsection{What the Results Do Not Establish}

The results do \emph{not} establish:

\begin{enumerate}
    \item That RH is true
    \item That the construction is mathematically equivalent to zeta
    \item That the bridge is unique or canonical
    \item That finite-size results extend to $N \to \infty$
\end{enumerate}

\subsection{Path Forward}

Future work could address:

\begin{enumerate}
    \item \textbf{Convergence analysis:} Study $N \to \infty$ systematically.
    \item \textbf{Bridge derivation:} Attempt to derive the mapping from first principles.
    \item \textbf{Additional properties:} Heat kernel, trace formula, spectral statistics.
    \item \textbf{Theoretical gap:} Identify what would logically connect
    the construction to zeta properties.
\end{enumerate}

% ============================================================================
% 10. CONCLUSION
% ============================================================================

\section{Conclusion}
\label{sec:conclusion}

We have presented a constraint-closure diagnostic framework and applied it
to an RH-motivated spectral case study. The main observations are:

\begin{enumerate}
    \item \textbf{Full closure:} All five levels (L0--L4) pass with residuals
    $< 2 \times 10^{-14}$, indicating machine-precision satisfaction of all
    constraint families within this construction.

    \item \textbf{Parameter robustness:} Results are stable across a $3 \times 3$
    parameter sweep in the tested range.

    \item \textbf{Bridge correspondence:} Projected spectrum correlates with
    reference data at $\rho = 0.9997$ (rank correlation).

    \item \textbf{Falsifiability:} Three negation tests (wrong ordering, wrong scale,
    wrong coordinate) all degrade results---ordering perturbation drops rank correlation
    from $\rho = 0.9997$ to $\rho = 0.008$, scale perturbation increases RMSE from 1098
    to 2395, and coordinate perturbation shifts $\beta$ deviation from 0 to 0.2.
\end{enumerate}

\begin{quote}
\textbf{This paper does not claim a proof of the Riemann Hypothesis.}

We present a diagnostic framework, empirical results, and falsifiability analysis.
The construction satisfies RH-motivated constraints and exhibits correspondence
with reference data that is inconsistent with simple null models.

Whether these observations can be developed into a rigorous mathematical
connection remains open.

The framework, data, and software are fully reproducible. We invite
independent verification and critique.
\end{quote}

% ============================================================================
% ACKNOWLEDGMENTS
% ============================================================================

\section*{Acknowledgments}

We thank the developers of NumPy, SciPy, and mpmath for numerical infrastructure.
All computations were performed on standard hardware with open-source software.

% ============================================================================
% APPENDICES
% ============================================================================

\appendix

\section{CLI Commands for Reproduction}
\label{app:commands}

\begin{lstlisting}
# Environment setup
python3 -m venv venv
source venv/bin/activate
pip install -e .

# Run used for this paper (seed 42)
rhdiag bundle --seed 42 --cell-count 64 --internal-dim 96 \
              --propagation-range 3 --outdir runs/

# Individual commands
rhdiag run --seed 42 --cell-count 64 --internal-dim 96
rhdiag run --seed 42 --cell-count 64 --internal-dim 96 --bridge-mode BA
rhdiag sweep --param1 cell_count --values1 16,32,64 \
             --param2 propagation_range --values2 1,2,3

# Verify determinism
rhdiag run --seed 42 && rhdiag run --seed 42
# Both produce identical hash
\end{lstlisting}

\section{Configuration Tables}
\label{app:config}

\begin{table}[H]
\centering
\caption{Production Run Configuration}
\begin{tabular}{@{}ll@{}}
\toprule
Parameter & Value \\
\midrule
\texttt{seed} & 42 \\
\texttt{cell\_count} & 64 \\
\texttt{internal\_dim} & 96 \\
\texttt{orbit\_size} & 12 \\
\texttt{orbit\_count} & 8 \\
\texttt{propagation\_range} & 3 \\
\texttt{bridge\_mode} & BA (for bridge run) \\
\bottomrule
\end{tabular}
\end{table}

\begin{table}[H]
\centering
\caption{Numerical Tolerances}
\begin{tabular}{@{}ll@{}}
\toprule
Constraint & Tolerance \\
\midrule
Closure ladder & $10^{-8}$ \\
Torsion order ($T^{12} = I$) & $10^{-12}$ \\
Weyl relation & $10^{-10}$ \\
Projector identities & $10^{-12}$ \\
Kernel non-negativity & $10^{-10}$ \\
\bottomrule
\end{tabular}
\end{table}

\section{Manifest Data}
\label{app:manifest}

Run hash: \texttt{85568e827299b531}

Timestamp: 2026-01-13T22:57:58

Package versions:
\begin{itemize}
    \item NumPy 1.24.4
    \item SciPy 1.10.1
    \item mpmath 1.3.0
\end{itemize}

Bridge metrics:
\begin{itemize}
    \item $n$ compared: 5000
    \item Correlation: 0.99994
    \item Rank correlation: 0.99973
    \item RMSE: 1098.0
    \item Mean $\beta$: 0.5
    \item $\beta$ deviation: 0.0
\end{itemize}

Falsification metrics:
\begin{itemize}
    \item BN1: Spearman $\rho$ drops from $0.9997$ to $0.008$ (ordering destroyed)
    \item BN2: RMSE increases from $1098$ to $2395$ ($2.2\times$ worse)
    \item BN3: $\beta$ deviation shifts from $0$ (exact) to $0.2$
\end{itemize}

\section{Heuristic Statistical Estimates}
\label{app:statistics}

\textbf{Disclaimer:} The following estimates are heuristic and intended for
scale intuition, not rigorous probabilistic bounds. They assume simplified
null models and independence between properties, which may not hold.

\subsection{Rank Correlation}

For $n = 200$ samples, under a null hypothesis of random ordering:
\begin{equation}
\rho \sim N\left(0, \frac{1}{\sqrt{n-1}}\right) = N(0, 0.0709)
\end{equation}

Observed $\rho = 0.9997$:
\begin{equation}
z = \frac{0.9997}{0.0709} \approx 14.1
\end{equation}

Under this simplified model, $P(Z > 14.1)$ is vanishingly small.

\subsection{Kernel Positivity}

For $n = 384$ eigenvalues, under a naive null hypothesis where each
eigenvalue has equal probability of being positive or negative:
\begin{equation}
P(\text{all positive}) = 0.5^{384}
\end{equation}

This is a simplified model; the actual construction guarantees positivity
by Proposition~\ref{prop:positivity}.

\subsection{Falsification Tests}

Under a null hypothesis where the mapping has no structural relationship
to reference data, the probability that all three negation tests would
produce worse results by the observed margins is small.

\subsection{Combined Estimate}

Under independence assumptions (which may not hold):
\begin{equation}
P_{\text{combined}} \ll 1
\end{equation}

These estimates suggest the observed correspondence is non-generic
under these simplified models. However, they should not be
interpreted as rigorous probability bounds due to the simplifying
assumptions involved.

% ============================================================================
% BIBLIOGRAPHY
% ============================================================================

\bibliographystyle{plainnat}
\begin{thebibliography}{9}

\bibitem[Odlyzko(1989)]{odlyzko1989}
A.~M. Odlyzko.
\newblock The $10^{20}$-th zero of the Riemann zeta function and 175 million of its neighbors.
\newblock Preprint, AT\&T Bell Laboratories, 1989.

\bibitem[Li(1997)]{li1997}
X.-J. Li.
\newblock The positivity of a sequence of numbers and the Riemann hypothesis.
\newblock \emph{J. Number Theory}, 65(2):325--333, 1997.

\bibitem[Montgomery(1973)]{montgomery1973}
H.~L. Montgomery.
\newblock The pair correlation of zeros of the zeta function.
\newblock \emph{Proc. Sympos. Pure Math.}, 24:181--193, 1973.

\bibitem[Conrey(2003)]{conrey2003}
J.~B. Conrey.
\newblock The Riemann hypothesis.
\newblock \emph{Notices of the AMS}, 50(3):341--353, 2003.

\bibitem[Bombieri(2000)]{bombieri2000}
E.~Bombieri.
\newblock The Riemann hypothesis.
\newblock In \emph{The Millennium Prize Problems}, pages 107--124. Clay Mathematics Institute, 2000.

\end{thebibliography}

\end{document}
